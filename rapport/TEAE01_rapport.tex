\documentclass[10pt,a4paper]{article} 
% Style
\usepackage{amsfonts}
\usepackage{amsmath}
\usepackage{amssymb}
\usepackage[utf8]{inputenc}
\usepackage[T1]{fontenc}
\usepackage[swedish]{babel}
\usepackage{lmodern} 
\usepackage{graphicx}
\usepackage{color}
\usepackage{float}
\usepackage{listings}
\lstset{extendedchars=\true}
\lstset{inputencoding=ansinew}

% New commands
\newcommand{\degree}{\ensuremath{^\circ}}




% Formler


% Sources

% ---Title page ---

\title{Företagsanalys Tekniska Verken}

\author{Johan Berneland, Sofia Larsson Cahlin, Benjamin Lundahl \\
    Nora Björklund, Christopher Hallberg}
\date{\today}

% ---Dokument start---

\begin{document}


\maketitle

\newpage

\tableofcontents

\newpage

\section{Bakgrund och syfte}
Bakgrund

\section{Metod}

\section{Resultat och analys}
\subsection{Företagsbeskrivning}
\subsubsection{Historik}
\subsubsection{Vision, mission, affärsidé}
Tekniska verkens vision är att '' bygga världens mest resurseffektiva
region'' och deras mission ''att tillhandahålla och utveckla
ledningsbunden infrastruktur och energilösningar för den
resurseffektiva regionen'' (Tekniska Verken ''ledning'', 2013). Det framgår även
från texter på deras hemsida att deras affärsidé är [FIXME].
\subsubsection{Ägarsituation/aktier}
Tekniska verken ägs av Linköping kommun (Tekniska verken, 2013). Deras
aktiekapital uppgår i 200 miljoner kronor och det finns 400 tusen
aktier (Retriever, 2013).


\section{Slutsatser}

\section{Referenser}


\end{document}



%%% Local Variables: 
%%% TeX-PDF-mode: t 
%%% TeX-master: "rapport"
%%% End: 

