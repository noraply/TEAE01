\documentclass[10pt,a4paper]{article} 
% Style
\usepackage{amsfonts}
\usepackage{amsmath}
\usepackage{amssymb}
\usepackage[utf8]{inputenc}
\usepackage[T1]{fontenc}
\usepackage[swedish]{babel}
\usepackage{lmodern} 
\usepackage{graphicx}
\usepackage{color}
\usepackage{float}
\usepackage{listings}
\lstset{extendedchars=\true}
\lstset{inputencoding=ansinew}

% New commands
\newcommand{\degree}{\ensuremath{^\circ}}




% Formler


% Sources

% ---Title page ---

\title{Företagsanalys Tekniska Verken}

\author{Johan Berneland, Sofia Larsson Cahlin, Benjamin Lundahl \\
    Nora Björklund, Christopher Hallberg}
\date{\today}

% ---Dokument start---

\begin{document}


\maketitle

\newpage

\tableofcontents

\newpage

\section{Bakgrund och syfte}
Bakgrund

\section{Metod}

\section{Resultat och analys}
\subsection{Företagsbeskrivning}
\subsubsection{Historia}
Grunden för Tekniska verken lades den 18 oktober 1902 genom startandet av
Linköpings elektriska kraft- och belysningsaktiebolag. Initiativtagare var en
entreprenör vid namn Jonn O Nilson och målet var att förse Linköpingsborna med
elström. År 1952 påbörjades fjärrvärmeutbyggnaden i Linköping och 2 år senare
kunde ett kraftvärmeverk leverera den första fjärrvärmen i Åbylund. Utbyggnaden
fortsatte och ytterligare kraftverk kom till under kommande årtionden. År 1981 
invigdes Gärstadsverket som utnyttjade avfall som bränsle. I samband med detta
började även utvecklingen att ta till vara på avfall för att producera biogas 
och biogödsel. (Tekniska verken, 2013) 


\subsubsection{Marknadsstrategi och Kritiska framgångsfaktorer}
%% Hittar ingen konkret fakta. Vi borde diskutera fram vad VI anser är deras
%% marknadsstrategi och kff istället


\section{Slutsatser}

\section{Referenser}
%% Här placeras referenser enligt Harvard metoden. Skall vara alfabetiskt, för
%% övrig info se harvard-lathunden.
%%
%% Vi bör bestämma om vi ska länka varje enskild liten del av hemsidan eller
%% bara referera till deras ``Om oss''
Tekniska verken, (2013). Om oss. (HTML) Tillgänglig: \newline
<http://www.tekniskaverken.se/om-oss/index.xml> (2013-11-20)

\end{document}



%%% Local Variables: 
%%% TeX-PDF-mode: t 
%%% TeX-master: "rapport"
%%% End: 

