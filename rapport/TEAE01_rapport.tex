\documentclass[10pt,a4paper]{article} 
% Style
\usepackage{amsfonts}
\usepackage{amsmath}
\usepackage{amssymb}
\usepackage[utf8]{inputenc}
\usepackage[T1]{fontenc}
\usepackage[swedish]{babel}
\usepackage{lmodern} 
\usepackage{graphicx}
\usepackage{color}
\usepackage{float}
\usepackage{listings}
\lstset{extendedchars=\true}
\lstset{inputencoding=ansinew}
\usepackage{hyperref}

% New commands
\newcommand{\degree}{\ensuremath{^\circ}}




% Formler


% Sources

% ---Title page ---

\title{Företagsanalys Tekniska Verken}

\author{Johan Berneland\\Sofia Larsson Cahlin\\Benjamin Lundahl\\
    Nora Björklund\\Christopher Hallberg}
\date{\today}

% ---Dokument start---

\begin{document}


\maketitle

\newpage

\tableofcontents

\newpage

\section{Bakgrund och syfte}
Bakgrund

\section{Metod}
Rapporten grundar sig på data hämtad från årsredovisningar från företaget vad
gäller de ekonomiska delarna. I de delar som rör företagets historia, vision,
mission och liknande har vi använt oss av information från företagets hemsida.

\section{Resultat och analys}
\subsection{Företagsbeskrivning}

\subsubsection{Historia}
Grunden för Tekniska verken lades den 18 oktober 1902 genom startandet av
Linköpings elektriska kraft- och belysningsaktiebolag. Initiativtagare var en
entreprenör vid namn Jonn O Nilson och målet var att förse Linköpingsborna med
elström. År 1952 påbörjades fjärrvärmeutbyggnaden i Linköping och 2 år senare
kunde ett kraftvärmeverk leverera den första fjärrvärmen i Åbylund. Utbyggnaden
fortsatte och ytterligare kraftverk kom till under kommande årtionden. År 1981 
invigdes Gärstadsverket som utnyttjade avfall som bränsle. I samband med detta
började även utvecklingen att ta till vara på avfall för att producera biogas 
och biogödsel. (Tekniska verken, 2013) 
\subsubsection{Vision, mission, affärsidé}
Tekniska verkens vision är att '' bygga världens mest resurseffektiva
region'' och deras mission är ''att tillhandahålla och utveckla
ledningsbunden infrastruktur och energilösningar för den
resurseffektiva regionen'' (Tekniska Verken ''ledning'', 2013). Det framgår även
från texter på deras hemsida att deras affärsidé är [FIXME].
\subsubsection{Ägarsituation/aktier}
Tekniska verken ägs av Linköping kommun (Tekniska verken, 2013). Deras
aktiekapital uppgår i 200 miljoner kronor och det finns 400 tusen
aktier.(Retriever, 2013)

\subsubsection{Erbjudande: produkter och/eller tjänster}
Tekniska verken erbjuder många tjänster inom många olika sektorer i sammhället.
\begin{itemize}
	\item \textbf{Energi ur avfall:} Genom att utnytja avfall som energikälla
	produceras fjärrvärme, fjärrkyla och el till det nordiska elnätet. Dessutom och
	som följd av detta återvinner, behandlar och deponerar man avfall från hushåll
	och industrier.
	\item \textbf{Vatten:} Företaget renar vatten från sjöar och vattendrag till
	dricksvatten, renar Avloppsvatten innan det släpps ut i naturen igen, sammlar
	upp nederbörd och kvalitetstestar vatten i ett laboratorium.
	\item \textbf{Infrastruktur:} Tekniska verken Driftum är en sammarbetspartner
	som tillsammans levererar kvalitativa konsulttjänster i områdena infrastruktur,
	mätteknik och energieffektivisering.
	\item \textbf{Nät:} Dotterbolaget Tekniska verken Linköping Nät AB är
	specialister på nät och förser regionen med elektricitet och bredband samt
	helhetslösningar för utomhusbelysning.
	\item \textbf{Bredband:} Tekniska verken är även delägare i bolaget Utsikt
	Bredband som är en av regionens ledande leverantörer av bredband till hushåll,
	företag, operatörer och fastighetsägare.
	\item \textbf{Miljövänlig el:} Teknsika verken är även delägare i bolaget Bixia
	AB, som köper in miljövänlig el från förnybara källor.
	\item \textbf{Biogas:} Dotterbolaget Svensk Biogas driver utveklingen av
	marknaden för biogas och arbetar med etablering av tankställen för allmänheten.
	Dessutom utvecklar man processer och produktionskoncept.
	\item \textbf{Forskning och utveckling:} Tekniska verken arbetar även med att
	förbättra och utveckla processer som gynnar miljön och ekonomin.
\end{itemize}

\subsubsection{Marknadsstrategi och Kritiska framgångsfaktorer}
%% Hittar ingen konkret fakta. Vi borde diskutera fram vad VI anser är deras
%% marknadsstrategi och kff istället

\subsection{Ekonomisk beskrivning}


\subsubsection{Nyckeltal}
%% En massa skit, TODO: städa upp
\begin{tabular}{ l r }
	Nyckeltal & 2012-12\\
	Antal anställda, aktiebolag & 969\\
	Avkastning på eget kapital (\%) & 14,16\\
	Avkastning på totalt kapital (\%) & 6,85\\
	Skuldränta (\%) & 2,19\\
	Riskbuffert totalt kapital & 4,66\\
	Rörelseresultat före avskrivningar, EBITDA & 1 052 000\\
	Rörelsemarginal (\%) & 10,64\\
	Vinstmarginal (\%) & 10,89\\
	Omsättning per anställd & 5 410,73\\
	Soliditet (\%) & 39,39\\
	Kapitalets omsättningshastighet & 0,63\\
	Rörelsekapital & 957 000\\
	Rörelsekapital/omsättning (\%) & 18,25\\
	Kassalikviditet (\%) & 159\\
	Förändring av omsättning (\%) & -7,27\\
	Rörelseresultat per anställd & 575,85\\
	Personalkostnader per anställd & 631,58\\
	Förändring av antal anställda (\%) & -0,31\\
	Lager mm/omsättning (\%) & 2,25\\
	Kundfordringar/omsättning (\%) & 10,41\\
	Likvida medel/omsättning (\%) & 11,81\\
	Kortfristiga skulder/omsättning (\%) & 27,12\\
	Skuldsättningsgrad (\%) & 1,48\\
	Räntetäckningsgrad (\%) & 5,39\\
	Avkastning operativt kapital (\%) & 8,86\\
	Riskbuffert sysselsatt kapital st & 5,29\\
	Varulagrets omsättningshastighet & 0\\
	Du Pont-modellen (\%) & 6,85\\
\end{tabular}
\subsubsection{Jämförelse över tid}
%% Vad ska jämföras? Nyckeltalen? Bara copy pasta från retriever?
%% Slutsatser/ren fakta?

\subsubsection{Jämförelse med konkurrenter}


\section{Slutsatser}

\section{Referenser}
%% Här placeras referenser enligt Harvard metoden. Skall vara alfabetiskt, för
%% övrig info se harvard-lathunden.
%%
%% Vi bör bestämma om vi ska länka varje enskild liten del av hemsidan eller
%% bara referera till deras ``Om oss''
Tekniska verken, (2013). Om oss. (HTML) Tillgänglig: \newline
<http://www.tekniskaverken.se/om-oss/index.xml> (2013-11-20)\\
Tekniska verken, (2013). Tjänster. (HTML) Tillgänglig: \\
\hyperref{http://www.tekniskaverken.se/tjanster/}{}{}{<http://www.tekniskaverken.se/tjanster/>(20-11-2013)}
\end{document}



%%% Local Variables: 
%%% TeX-PDF-mode: t 
%%% TeX-master: "rapport"
%%% End: 

